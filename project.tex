\section{Progetto}
\label{project}
Il progetto è composto da 5 sezioni:
\begin{itemize}
\item\emph{graph}
\item\emph{distance}
\item\emph{movielens\_exp}
\item\emph{profile}
\item\emph{recommendation}
\end{itemize}
\subsection{graph}
Nella parte iniziale del progetto verranno eseguiti dei metodi presenti nella classi situate all'interno del package \emph{graph}, i cui scopi saranno quelli di: 
\begin{itemize}
\item Salvare all'interno di un grafo i film presenti nel db di movielens;
\item Partendo dal grafo creato in precedenza, viene creato un nuovo grafo in cui ogni vertice rappresenta un film, mentre ogni arco...
\end{itemize}
\subsection{distance}
\subsection{movielens\_exp}
\subsection{profile}
\subsection{recommendation}


Il compito principale di questo modulo consiste inizialmente nella creazione dell'oggetto graph, il cui scopo è quello di contenere in un multigrafo sparso non direzionato le varie risorse estratte dal mapping dei film contenuti nel db di movielens con dbPedia

