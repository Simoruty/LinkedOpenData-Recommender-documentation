\section{Misure}

\subsection{Distanze}

Il nostro recommender system è stato testato su quattro diverse misure di distanza.
Le prime due sono descritte in \citet{passant2010measuring}, le altre due è stata realizzata \emph{ex novo} e rappresentano una nuova visione pesata delle precedenti.


\subsubsection{Passant Direct}
\label{PassantD}
La distanza diretta. Indica il numero di archi diretti che collegano i film $a$ e $b$. Tenendo conto che il grafo preso in considerazione è orientato, vanno presi gli archi in entrambe le direzioni.

In maniera formale, definita $$C_{d}(f_a,f_b) = \left\vert \left\{ e \in Edges \  | \  (f_a \xrightarrow{~e~} f_b ) \right\} \right\vert$$ la formula della distanza completa sarà quindi:

%\begin{figure}[htbp]
%  \centering
%    \label{LDSD}
%    \begin{equation}
%        P_{d}(r_{a},r_{b}) = \frac{1} {1+C_{d}(n,r_{a},r_{b})+C_{d}(n,r_{b},r_{a})}
%    \end{equation}
%      \caption{LDSD Distance}
%      \label{LDSD1}
%\end{figure}

    \begin{equation}
        P_{d}(f_{a},f_{b}) = \frac{1} {1+C_{d}(f_{a},f_{b})+C_{d}(f_{b},f_{a})}
    \end{equation}

\subsubsection{Passant Combinated}
\label{PassantC}

La distanza combinata compone la precedente distanza diretta con una misura che
prende in considerazione gli archi che collegano due film  secondo la formula:

Indica la distanza indiretta trai film presi in esame; il valore restituito sta
ad indicare il numero di archi che mettono in relazione i due film passanti attraverso dei film in comune.

Formalmente, definiti:
$$C_{io}(f_a,f_b) = \left\vert \left\{ o \  | \  (f_a \xrightarrow{~e~} o ) \wedge (f_b \xrightarrow{~e~} o) \right\} \right\vert$$
$$C_{ii}(f_a,f_b) = \left\vert \left\{ s \  | \  ( s \xrightarrow{~e~} f_a ) \wedge ( s \xrightarrow{~e~} f_b) \right\} \right\vert$$ 
$$\text{con} \ e \in Edges  , \qquad f_a,f_b,o,s \in Films $$

la formula indiretta di Passant è:
    \begin{equation}
        P_i(f_{a},f_{b}) = \frac{1} {1+C_{io}(f_{a},f_{b})+C_{ii}(f_{a},f_{b})}
    \end{equation}



    \begin{equation}
P_{c}(f_{a},f_{b}) = \frac{1} {1+C_{d}(f_{a},f_{b})+C_{d}(f_{b},f_{a})+C_{io}(f_{a},f_{b})+C_{ii}(f_{a},f_{b})}
    \end{equation}


\subsubsection{Passant Direct Weighted}
La nostra distanza prende in considerazione sia gli archi diretti che gli archi indiretti (quelli che uniscono i film $a$ e $b$ passando da un oggetto o soggetto comune, come descritto in \ref{PassantC}).

La distanza è calcolata in base ai pesi di ogni arco presente sul \emph{FilmGraph}.

In maniera formale: 

\begin{eqnarray*}
ED_{f_a,f_b} = & \{ e \in Edges \ \quad | \, & \  s(a) = f_a \wedge t(a) = f_b \} \\
EIO_{f_a,f_b} = & \{ e \in Edges \ \quad | \, & \  s(a) = f_a \wedge s(b) = f_b \wedge t(a) = t(b) \}  \\
EII_{f_a,f_b} = & \{ e \in Edges \ \quad | \, & \  s(a) = f_a \wedge s(b) = f_b \wedge t(a) = t(b) \}  \\
\end{eqnarray*}

\begin{eqnarray}
N_{d}(f_a,f_b) = \sum\limits_{e \in A}^{}{w(e)} \\
N_{io}(f_a,f_b) = \sum\limits_{e \in B}^{}{w(e)} \\
N_{ii}(f_a,f_b) = \sum\limits_{e \in C}^{}{w(e)} \\ 
\end{eqnarray}


\subsubsection{Passant Combinated Weighted}

\lipsum



\subsection{Profili}

\subsubsection{Simple}
Il profilo coincide con l'insieme dei film valutati positivamente (4 e 5 considerati in egual misura)

\subsubsection{Simple Negative}

I film valutati 4 e 5 sono nell'insieme dei positivi. I film valutati 1 e 2 sono nell'insieme dei negativi.


\subsubsection{Voted Nostra}


Il profilo è l'insieme dei film visti, pesati secondo la formula:
votazioni dei film visti - voto medio dei film.

\subsubsection{Voted Musto}

Il profilo è l'insieme dei film visti, pesati secondo la formula:
Voto massimo + 1 - voto del film visto. 

