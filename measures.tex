\section{Misure}

\subsection{Distanze}

Il nostro recommender system è stato testato su quattro diverse misure di distanza.
Le prime tre sono descritte in \citet{passant2010measuring}, la quarta è stata realizzata \emph{ex novo}.



\subsubsection{Passant Direct}
La distanza diretta. Indica il numero di archi diretti che collegano i film $a$ e $b$. Tenendo conto che il grafo preso in considerazione è orientato, vanno presi gli archi in entrambe le direzioni.

In maniera formale, definita $$C_{d}(f_a,f_b) = \left\Big\vert \left\Big\{ e \in Edges \  \big| \  (f_a \xrightarrow{~e~} f_b ) \right\} \right\vert$$ la formula della distanza completa sarà quindi:

%\begin{figure}[htbp]
%  \centering
%    \label{LDSD}
%    \begin{equation}
%        P_{d}(r_{a},r_{b}) = \frac{1} {1+C_{d}(n,r_{a},r_{b})+C_{d}(n,r_{b},r_{a})}
%    \end{equation}
%      \caption{LDSD Distance}
%      \label{LDSD1}
%\end{figure}

    \begin{equation}
        P_{d}(f_{a},f_{b}) = \frac{1} {1+C_{d}(f_{a},f_{b})+C_{d}(f_{b},f_{a})}
    \end{equation}



\subsubsection{Passant Indirect}

Indica la distanza indiretta trai film presi in esame; il valore restituito sta ad indicare il numero di archi che mettono in relazione i due film passanti attraverso dei film in comune.

Formalmente, definiti:
$$C_{io}(f_a,f_b) = \left\Big\vert \left\Big\{ o \  \big| \  (f_a \xrightarrow{~e~} o ) \wedge (f_b \xrightarrow{~e~} o) \right\} \right\vert$$
$$C_{ii}(f_a,f_b) = \left\Big\vert \left\Big\{ s \  \big| \  ( s \xrightarrow{~e~} f_a ) \wedge ( s \xrightarrow{~e~} f_b) \right\} \right\vert$$ 
$$\text{con} \ e \in Edges  , \qquad f_a,f_b,o,s \in Films $$

la formula indiretta di Passant è:
    \begin{equation}
        P_{i}(f_{a},f_{b}) = \frac{1} {1+C_{io}(f_{a},f_{b})+C_{ii}(f_{b},f_{a})}
    \end{equation}


\subsubsection{Passant Combinated}
Una distanza che combina le due distanze precedenti secondo la formula:

    \begin{equation}
P_{c}(f_{a},f_{b}) = \frac{1} {1+C_{io}(f_{a},f_{b})+C_{ii}(f_{b},f_{a})}
    \end{equation}


Nostra
Si prendono in considerazione sia gli archi diretti che uniscono a e b, si gli archi indiretti, cioè quelli che legano a e b ad un oggetto o soggetto comune.
Un link è un collegamento diretto tra due film con un suo peso dato dalla produttoria dei pesi degli archi precedentemente trovati.
La distanza coincide con la sommatoria del peso di ogni link.



\subsection{Profili}

\subsubsection{Simple}
Il profilo coincide con l'insieme dei film valutati positivamente (4 e 5 considerati in egual misura)

\subsubsection{Simple Negative}

I film valutati 4 e 5 sono nell'insieme dei positivi. I film valutati 1 e 2 sono nell'insieme dei negativi.


\subsubsection{Voted Nostra}


Il profilo è l'insieme dei film visti, pesati secondo la formula:
votazioni dei film visti - voto medio dei film.

\subsubsection{Voted Musto}

Il profilo è l'insieme dei film visti, pesati secondo la formula:
Voto massimo + 1 - voto del film visto. 

