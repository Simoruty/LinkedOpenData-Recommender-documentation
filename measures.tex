\section{Misure}

\subsection{Distanze}

Il nostro recommender system è stato testato su quattro diverse misure di distanza.
Le prime tre sono descritte in \citet{passant2010measuring}, la quarta è stata realizzata \emph{ex novo}.



\subsubsection{Passant Direct}
La distanza diretta. Indica il numero di archi diretti che collegano i film $a$ e $b$. Tenendo conto che il grafo preso in considerazione è orientato, vanno presi gli archi in entrambe le direzioni.

In maniera formale, $C_d(n,r_a,r_b)$ restituisce il numero di archi che vanno dal film $a$ al film $b$. 
$C_d(n,r_b,r_a)$ restituisce il numero di archi che vanno dal film $b$ al film $a$.

La formula della distanza completa sarà quindi:

%\begin{figure}[htbp]
%  \centering
%    \label{LDSD}
    \begin{equation}
        P_{d}(r_{a},r_{b}) = \frac{1} {1+C_{d}(n,r_{a},r_{b})+C_{d}(n,r_{b},r_{a})}
    \end{equation}
%      \caption{LDSD Distance}
%      \label{LDSD1}
%\end{figure}




\subsubsection{Passant Indirect}

Indica la distanza indiretta trai film presi in esame; il valore restituito sta ad indicare il numero di archi che mettono in relazione i due film passanti attraverso delle risorse condivise.

Formalmente, dati:

$$C_{io}(n,r_a,r_b) = \left\vert \left\{ o \  | \  (r_a \  p \  o ) \wedge (r_b \  p \  o) \right\} \right\vert \text{, con} \ p \in P$$
$$C_{ii}(n,r_a,r_b) = \left\vert \left\{ s \  | \  ( s \  p \  r_a ) \wedge ( s \  p \  r_b) \right\} \right\vert \text{,~con} \  p \in P$$ 

la formula indiretta di Passant è:

    \begin{equation}
        P_{i}(r_{a},r_{b}) = \frac{1} {1+C_{io}(n,r_{a},r_{b})+C_{ii}(n,r_{b},r_{a})}
    \end{equation}

indica il numero di archi che collegano indirettamendata una coppia di triple RDF (soggetto–predicato–oggetto), si sommano gli archi contenuti nelle triple RDF aventi come soggetti i due film in questione e come oggetto un film in comune ad entrambe le triple RDF.
$r_a$ – predicato – $o_1$
$r_b$ – predicato – $0_1$
Cii(n,ra,rb) indica il numero di archi che soddisfano la seguente proprietà:
data una tripla RDF: soggetto – predicato – oggetto, si sommano gli archi contenuti nelle triple rdf aventi come oggetti i due film in questione e come soggetto un film in comune ad entrambi le triple RDF.
oggetto1 – predicato – ra
oggetto1 – predicato – rb



\subsubsection{Passant Combinated}
Indica la distanza combinata tra i due film in questione utilizzando sia la distanza indiretta, che quella diretta.

Nostra
Si prendono in considerazione sia gli archi diretti che uniscono a e b, si gli archi indiretti, cioè quelli che legano a e b ad un oggetto o soggetto comune.
Un link è un collegamento diretto tra due film con un suo peso dato dalla produttoria dei pesi degli archi precedentemente trovati.
La distanza coincide con la sommatoria del peso di ogni link.



\subsection{Profili}

\subsubsection{Simple}
Il profilo coincide con l'insieme dei film valutati positivamente (4 e 5 considerati in egual misura)

\subsubsection{Simple Negative}

I film valutati 4 e 5 sono nell'insieme dei positivi. I film valutati 1 e 2 sono nell'insieme dei negativi.


\subsubsection{Voted Nostra}


Il profilo è l'insieme dei film visti, pesati secondo la formula:
votazioni dei film visti - voto medio dei film.

\subsubsection{Voted Musto}

Il profilo è l'insieme dei film visti, pesati secondo la formula:
Voto massimo + 1 - voto del film visto. 

