\section{Sperimentazione}
\label{experiment}


\subsection{Metriche}
\begin{itemize}
\item Precision
\item MRR
\end{itemize}
\subsubsection{Precision}
\textbf{la precisione è definita come il numero di documenti attinenti recuperati da una ricerca diviso il numero totale di documenti recuperati dalla stessa ricerca,
la precisione è la frazione di documenti attinenti che sono stati trovati
In un processo di classificazione statistica, la precisione per una classe è il numero di veri positivi (il numero di oggetti etichettati correttamente come appartenenti alla classe) diviso il numero totale di elementi etichettati come appartenenti alla classe (la somma di veri positivi e falsi positivi, che sono oggetti etichettati erroneamente come appartenenti alla classe).
In un processo di classificazione, i termini vero positivo, vero negativo, falso positivo e falso negativo sono usati per confrontare la classificazione di un oggetto (l’etichetta di classe assegnata all’oggetto da un classificatore) con la corretta classificazione desiderata (la classe a cui in realtà appartiene l’oggetto).}
$$
Precision =\frac{T_P}{T_P+F_P}
$$
In particolare, nella fase sperimentale del progetto sono state calcolate 4 tipologie di precisioni:
\begin{itemize}
\item microprecision
\item microprecision epurata
\item macroprecision
\item macroprecision epurata
\end{itemize}
Le versioni epurate delle precisioni calcolate vanno
\subsubsection{MRR}
Il Mean Reciprocal Rank (MRR) è un indice statistico per valutare un processo che produce una lista di possibili risposte ad una interrogazione (query), ordinate per probabilità di correttezza.
$$
\begin{equation}
MRR = \frac{1}{|Q|}\sum_{i=1}^{Q}{\frac{1}{Rank_i}}
\end{equation}
$$

